\begin{abstract}
Ziel dieser Arbeit ist es eine Schnittstelle zwischen MATLAB und einer Wetterstation aufzubauen, um darüber Prognosedaten für ein integriertes Energiemanagementsystem bereitzustellen. Diese Informationen sollen dazu dienen die Planungen des Managementsystems im Smart-Micro-Grid hinsichtlich Lastverläufe und Energieerzeugung zu vereinfachen bzw. zu präzisieren. Die Datenbereitstellung erfolgt über einen Langwellenempfänger, dessen Register über eine MODBUS Kommunikation abgerufen werden können. Die meisten gelieferten Werte weisen eine zeitliche Auflösung von 6 Stunden auf. Da das Managementsystem jedoch umso genauer arbeiten kann, je niedriger diese Auflösung ist, ist es mit Aufgabe der Schnittstelle, die Daten in kleineren Zeitintervallen zur Verfügung zu stellen. Der Datenabruf und die Verarbeitung sollen in anderen MATLAB Programmen zum Einsatz kommen. Es ist daher zweckmäßig den Kommunikationsprozess als MATLAB Funktion mit entsprechenden Übergabeparametern zu implementieren. Um die geforderten Aufgabenziele zu erreichen, wurden die Spezifikationen der Wetterstation und des MODBUS-Protokolls analysiert. Mit den aus der Analyse gewonnen Informationen und den in MATLAB zur Verfügung stehenden Methoden, wurde letztlich das Programm umgesetzt. Wie der Leser am Ende der Arbeit feststellen kann, ergibt ein Vergleich der interpolierten Daten mit genauen Wetteraufzeichnungen der LMU ein differenziertes Bild. Eine ausführlichere Datenanalyse könnte in diesem Zusammenhang mehr Aufschluss geben. Doch war dies nicht der Schwerpunkt dieser Arbeit.        
\end{abstract}