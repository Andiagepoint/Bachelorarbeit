Aufgabe war es die Daten einer Wetterstation wiederholt auszulesen und für die weitere Verarbeitung zu interpolieren bzw. abzuspeichern. Für die Entwicklung des Programmcodes waren das Wissen um die MODBUS- und Wetterstationsspezifikation von entscheidender Bedeutung. Ausgehend von diesen Informationen konnte in einem ersten Schritt in einer virtuellen Umgebung die Kommunikation über die serielle Schnittstelle getestet werden. Mit den nun bekannten Funktionalitäten konnte der weitere Programmaufbau für das mehrmalige Senden und Auslesen von MODBUS-Nachrichten implementiert werden. Als kleine Schwierigkeit gestaltete sich der Aufbau der kontinuierlichen Zeitreihe, da die Grunddaten selbst in einer unterschiedlichen Auflösung vorliegen. Aber auch dieses Problem konnte letztendlich gelöst werden. Nachdem die ersten Ergebnisse vorlagen und mit den Daten des metereologischen Instituts der LMU verglichen wurden, ergaben sich neue Verbesserungsmöglichkeiten. Im Bereich der Solarleistung wurden daraufhin die Daten der Wetterstation auf den Tagesgang der Sonne gemappt. Ein nachträglicher Vergleich bestätigte den Erfolg dieser Maßnahme deutlich. Trotz der guten Datenaufbereitung gibt es noch Verbesserungspotential. Bezogen auf die Temperaturen kann aufbauend auf historische Daten, die vom lokalen Temperatursensor stammen, ein Ausgleichsfaktor bestimmt werden, welcher die Prognosedaten an die tatsächlich vor Ort vorherrschenden Temperaturen anpasst. Für einen Test über einen längeren Zeitraum als einer Woche war in dieser Arbeit leider keine Zeit. Daher wäre es zu empfehlen diesen noch durchzuführen. Der in dieser Arbeit ausgegebene Luftdruck bezieht sich auf das Niveau des Meeresspiegels. Soll ein ortsabhängiger Luftdruck in den Daten gespeichert werden, so müssen die jeweiligen Höhen der Wetterregionen und die barometrische Höhenformel mit in den Programmcode integriert werden. Die Sonnenauf- und -untergangsadaption ist bis jetzt nur für deutsche Städte vorgesehen. Sollen andere europäische Wetterregionen abgerufen werden, so müssen die Längen- und Breitengrade im Code ergänzt werden.     Da die Wetterstation dazu verwendet werden soll dem Energiemanagementsystem Prognosedaten für die Energiegewinnung und den Energieverbrauch zu liefern, ist es angesichts der momentanen Datenlage eine Frage, ob diese hierfür geeignet erscheint. Eine Alternative könnten Daten aus dem Internet darstellen, die ebenfalls kostenlos zur Verfügung gestellt werden. Jedoch liegen diese nicht wie in der Wetterstation als gesammeltes Paket vor und auch die Auflösung ist meist nicht einheitlich. So gesehen ist sie angesichts des Kosten-Leistungs-Verhältnisses eine ganz gute Lösung.   