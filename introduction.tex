In Zukunft wird die Mobilität durch Elektroautos mit geprägt sein. Damit Deutschland auf diesem Technologiefeld eine Spitzenposition einnehmen kann, wurde von der Bundesregierung die Nationale Plattform Elektromobilität initiiert. Ziel dieser Institution ist es Deutschland bis zum Jahr 2020 zum Leitmarkt und Leitanbieter zu entwickeln. Marktvorbereitung, Markthochlauf und der Massenmarkt sind dabei die zu durchlaufenden Phasen. In der Marktvorbereitungsphase, in der wir uns zur Zeit befinden, werden die Ergebnisse aus Forschung und Entwicklung genutzt, um in vier sogenannten Schaufenstern die Modelle und Prognosen für den Markthochlauf zu validieren bzw. bei auftretenden Abweichungen anzupassen.\cite{NPE}Eines dieser Schaufenster, genannt "{}Elektromobilität verbindet"{} wird von den Bundesländern Bayern und Sachsen betreut und finanziert. Das Schaufenster ist aufgegliedert in vier Teilprojekte von denen eines sich den Energiesystemen widmet. Das Themengebiet Energiesysteme ist wiederum in 9 Aufgabengebiete unterteilt, wovon sich eines mit der Integration der Elektromobilität in die dezentrale regenerative Energieversorung beschäftigt. Ein Aufgabenschwerpunkt hierbei ist es ein integriertes Energiemanagementsystem mittels Aufbau und Betrieb eines Hardware-in-the-Loop Prüfstands zu evaluieren. Da das Energiemanagementsystem auch Vorausschautechnologien einbinden soll, ist es erforderlich Prognosedaten zu erheben.\cite{SEEV}     