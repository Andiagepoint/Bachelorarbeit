Schenkt man der Studie von \enquote{Global EV Outlook} glauben, so wird die Mobilität in Zukunft durch elektrisch angetriebene Autos mit geprägt sein \cite{EVOutlook}. Damit Deutschland auf diesem Technologiefeld eine Spitzenposition einnehmen kann, wurde von der Bundesregierung die Nationale Plattform Elektromobilität initiiert. Ziel dieser Institution ist es Deutschland bis zum Jahr 2020 zum Leitmarkt und Leitanbieter zu entwickeln. Marktvorbereitung, Markthochlauf und der Massenmarkt sind dabei die zu durchlaufenden Phasen. In der Marktvorbereitungsphase, in der wir uns zur Zeit befinden, werden die Ergebnisse aus Forschung und Entwicklung genutzt, um in vier sogenannten Schaufenstern die Modelle und Prognosen für den Markthochlauf zu validieren bzw. bei auftretenden Abweichungen anzupassen \cite{NPE}. Eines dieser Schaufenster, genannt "{}Elektromobilität verbindet"{} wird von den Bundesländern Bayern und Sachsen betreut und finanziert. Das Schaufenster ist aufgegliedert in vier Teilprojekte von denen eines sich den Energiesystemen widmet. Das Themengebiet Energiesysteme ist wiederum in 9 Aufgabengebiete unterteilt, wovon sich eines mit der Integration der Elektromobilität in die dezentrale regenerative Energieversorung beschäftigt. Ein Aufgabenschwerpunkt hierbei ist es ein integriertes Energiemanagementsystem mittels Aufbau und Betrieb eines Hardware-in-the-Loop Prüfstands zu evaluieren \cite{SEEV}. Da es sich hierbei um ein Einfamilienhaus handelt spricht man von einem Home Energiemanagementsystem (HEMS). Was ist die Aufgabe eines solchen Systems und was macht ein solches System aus? Eine Antwort auf diese Fragen gibt der Artikel \enquote{Link to future} auf den sich die nachfolgenden Angaben beziehen \cite{LtoF}. Hier ist es Aufgabe eines HEMSs den Nutzer mit umfassenden Funktionen zum internen Informationsaustausch zu versorgen. Diese Informationen dienen letztendlich dazu den täglichen Energieverbrauch zu optimieren, um dadurch bei gleichbleibender Lebensqualität Kosten zu sparen. Drei Bausteine bilden dabei das Grundgerüst für das HEMS. 
\begin{enumerate}
\item Das \enquote{Energy Management Gateway} übernimmt dabei die Aufgabe des sicheren Datenaustauschs zwischen den hausinternen Gerätschaften sowie über das Versorgungsnetz zu den Energieversorgern. 
\item Die \enquote{Energy Management Unit (EMU)} sammelt alle Daten über den Energieverbrauch, die Energieerzeugung sowie -speicherung in einem Haushalt. Abgeleitet aus diesen Informationen und den momentanen Preisen für Energieverbrauch und -erzeugung regelt sie den Einsatz der Geräte. 
\item Die Bereitstellung von Informationen für die EMU erledigt ein Netzwerk von Sensoren und Microcontrollern. Hierzu wird häufig ein Home Area Network entweder drahtgebunden oder über Funk installiert.     
\end{enumerate} 
Diese Arbeit orientiert sich am dritten Baustein und soll Wetterdaten, einer später extern am Haus angebrachten Wetterstation, der EMU zur Verfügung stellen. Die EMU wird hierbei durch ein MATLAB-Programm dargestellt. 