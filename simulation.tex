Nachdem nun die MODBUS und Wetterstationsspezifikation analysiert wurden, kann man mit der Umsetzung des MATLAB Programmes beginnen. Für den Fall, dass man noch nie in MATLAB mit einer seriellen Schnittstelle gearbeitet hat, bietet es sich an, diese zunächst einmal zu simulieren. Hierzu wurden in dieser Arbeit zwei Programme verwendet. Das eine Programm simuliert den COM Port des MODBUS Slave, heißt \enquote{Virtual Serial Ports Emulator} und wird von \enquote{eterlogic.com} zum Download angeboten. Das andere Programm virtualisiert den MODBUS Slave selbst und heißt \enquote{PeakHMI MODBUS Serial RTU slave}. Anbieter hierfür ist die Firma Everest Software LLC. Der große Vorteil in der Simulation liegt darin, dass man in das virtualisierte Holdingregister an eine bestimmte Adresse Werte schreiben kann. Mit den Methoden von MATLAB kann man nun versuchen diesen Wert auszulesen. Mit dieser Methodik kann man schnell die Funktionalität des erarbeiteten Codes auch offline erproben.   